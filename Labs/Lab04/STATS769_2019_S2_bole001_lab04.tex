\documentclass[]{article}
\usepackage{lmodern}
\usepackage{amssymb,amsmath}
\usepackage{ifxetex,ifluatex}
\usepackage{fixltx2e} % provides \textsubscript
\ifnum 0\ifxetex 1\fi\ifluatex 1\fi=0 % if pdftex
  \usepackage[T1]{fontenc}
  \usepackage[utf8]{inputenc}
\else % if luatex or xelatex
  \ifxetex
    \usepackage{mathspec}
  \else
    \usepackage{fontspec}
  \fi
  \defaultfontfeatures{Ligatures=TeX,Scale=MatchLowercase}
\fi
% use upquote if available, for straight quotes in verbatim environments
\IfFileExists{upquote.sty}{\usepackage{upquote}}{}
% use microtype if available
\IfFileExists{microtype.sty}{%
\usepackage{microtype}
\UseMicrotypeSet[protrusion]{basicmath} % disable protrusion for tt fonts
}{}
\usepackage[margin=1in]{geometry}
\usepackage{hyperref}
\hypersetup{unicode=true,
            pdftitle={STATS 769 - Lab 04 - bole001},
            pdfauthor={Bernard O'Leary},
            pdfborder={0 0 0},
            breaklinks=true}
\urlstyle{same}  % don't use monospace font for urls
\usepackage{color}
\usepackage{fancyvrb}
\newcommand{\VerbBar}{|}
\newcommand{\VERB}{\Verb[commandchars=\\\{\}]}
\DefineVerbatimEnvironment{Highlighting}{Verbatim}{commandchars=\\\{\}}
% Add ',fontsize=\small' for more characters per line
\usepackage{framed}
\definecolor{shadecolor}{RGB}{248,248,248}
\newenvironment{Shaded}{\begin{snugshade}}{\end{snugshade}}
\newcommand{\KeywordTok}[1]{\textcolor[rgb]{0.13,0.29,0.53}{\textbf{#1}}}
\newcommand{\DataTypeTok}[1]{\textcolor[rgb]{0.13,0.29,0.53}{#1}}
\newcommand{\DecValTok}[1]{\textcolor[rgb]{0.00,0.00,0.81}{#1}}
\newcommand{\BaseNTok}[1]{\textcolor[rgb]{0.00,0.00,0.81}{#1}}
\newcommand{\FloatTok}[1]{\textcolor[rgb]{0.00,0.00,0.81}{#1}}
\newcommand{\ConstantTok}[1]{\textcolor[rgb]{0.00,0.00,0.00}{#1}}
\newcommand{\CharTok}[1]{\textcolor[rgb]{0.31,0.60,0.02}{#1}}
\newcommand{\SpecialCharTok}[1]{\textcolor[rgb]{0.00,0.00,0.00}{#1}}
\newcommand{\StringTok}[1]{\textcolor[rgb]{0.31,0.60,0.02}{#1}}
\newcommand{\VerbatimStringTok}[1]{\textcolor[rgb]{0.31,0.60,0.02}{#1}}
\newcommand{\SpecialStringTok}[1]{\textcolor[rgb]{0.31,0.60,0.02}{#1}}
\newcommand{\ImportTok}[1]{#1}
\newcommand{\CommentTok}[1]{\textcolor[rgb]{0.56,0.35,0.01}{\textit{#1}}}
\newcommand{\DocumentationTok}[1]{\textcolor[rgb]{0.56,0.35,0.01}{\textbf{\textit{#1}}}}
\newcommand{\AnnotationTok}[1]{\textcolor[rgb]{0.56,0.35,0.01}{\textbf{\textit{#1}}}}
\newcommand{\CommentVarTok}[1]{\textcolor[rgb]{0.56,0.35,0.01}{\textbf{\textit{#1}}}}
\newcommand{\OtherTok}[1]{\textcolor[rgb]{0.56,0.35,0.01}{#1}}
\newcommand{\FunctionTok}[1]{\textcolor[rgb]{0.00,0.00,0.00}{#1}}
\newcommand{\VariableTok}[1]{\textcolor[rgb]{0.00,0.00,0.00}{#1}}
\newcommand{\ControlFlowTok}[1]{\textcolor[rgb]{0.13,0.29,0.53}{\textbf{#1}}}
\newcommand{\OperatorTok}[1]{\textcolor[rgb]{0.81,0.36,0.00}{\textbf{#1}}}
\newcommand{\BuiltInTok}[1]{#1}
\newcommand{\ExtensionTok}[1]{#1}
\newcommand{\PreprocessorTok}[1]{\textcolor[rgb]{0.56,0.35,0.01}{\textit{#1}}}
\newcommand{\AttributeTok}[1]{\textcolor[rgb]{0.77,0.63,0.00}{#1}}
\newcommand{\RegionMarkerTok}[1]{#1}
\newcommand{\InformationTok}[1]{\textcolor[rgb]{0.56,0.35,0.01}{\textbf{\textit{#1}}}}
\newcommand{\WarningTok}[1]{\textcolor[rgb]{0.56,0.35,0.01}{\textbf{\textit{#1}}}}
\newcommand{\AlertTok}[1]{\textcolor[rgb]{0.94,0.16,0.16}{#1}}
\newcommand{\ErrorTok}[1]{\textcolor[rgb]{0.64,0.00,0.00}{\textbf{#1}}}
\newcommand{\NormalTok}[1]{#1}
\usepackage{graphicx,grffile}
\makeatletter
\def\maxwidth{\ifdim\Gin@nat@width>\linewidth\linewidth\else\Gin@nat@width\fi}
\def\maxheight{\ifdim\Gin@nat@height>\textheight\textheight\else\Gin@nat@height\fi}
\makeatother
% Scale images if necessary, so that they will not overflow the page
% margins by default, and it is still possible to overwrite the defaults
% using explicit options in \includegraphics[width, height, ...]{}
\setkeys{Gin}{width=\maxwidth,height=\maxheight,keepaspectratio}
\IfFileExists{parskip.sty}{%
\usepackage{parskip}
}{% else
\setlength{\parindent}{0pt}
\setlength{\parskip}{6pt plus 2pt minus 1pt}
}
\setlength{\emergencystretch}{3em}  % prevent overfull lines
\providecommand{\tightlist}{%
  \setlength{\itemsep}{0pt}\setlength{\parskip}{0pt}}
\setcounter{secnumdepth}{0}
% Redefines (sub)paragraphs to behave more like sections
\ifx\paragraph\undefined\else
\let\oldparagraph\paragraph
\renewcommand{\paragraph}[1]{\oldparagraph{#1}\mbox{}}
\fi
\ifx\subparagraph\undefined\else
\let\oldsubparagraph\subparagraph
\renewcommand{\subparagraph}[1]{\oldsubparagraph{#1}\mbox{}}
\fi

%%% Use protect on footnotes to avoid problems with footnotes in titles
\let\rmarkdownfootnote\footnote%
\def\footnote{\protect\rmarkdownfootnote}

%%% Change title format to be more compact
\usepackage{titling}

% Create subtitle command for use in maketitle
\providecommand{\subtitle}[1]{
  \posttitle{
    \begin{center}\large#1\end{center}
    }
}

\setlength{\droptitle}{-2em}

  \title{STATS 769 - Lab 04 - bole001}
    \pretitle{\vspace{\droptitle}\centering\huge}
  \posttitle{\par}
    \author{Bernard O'Leary}
    \preauthor{\centering\large\emph}
  \postauthor{\par}
      \predate{\centering\large\emph}
  \postdate{\par}
    \date{26 August 2019}


\begin{document}
\maketitle

\section{Describe the methods used to import the data to
R.}\label{describe-the-methods-used-to-import-the-data-to-r.}

\subsection{JSON data}\label{json-data}

\begin{Shaded}
\begin{Highlighting}[]
\KeywordTok{library}\NormalTok{(jsonlite)}
\end{Highlighting}
\end{Shaded}

\begin{verbatim}
## Warning: package 'jsonlite' was built under R version 3.6.1
\end{verbatim}

\begin{Shaded}
\begin{Highlighting}[]
\NormalTok{readJsonFileIntoDataframe <-}\StringTok{ }\ControlFlowTok{function}\NormalTok{(file_name) \{}
  \CommentTok{# Read in data}
  \KeywordTok{fromJSON}\NormalTok{(}\KeywordTok{readLines}\NormalTok{(file_name))}
\NormalTok{\}}


\NormalTok{file_numbers <-}\StringTok{ }\KeywordTok{c}\NormalTok{(}\DecValTok{1}\OperatorTok{:}\DecValTok{10}\NormalTok{)}
\NormalTok{file_names <-}\StringTok{ }\KeywordTok{paste0}\NormalTok{(}\StringTok{"trips-"}\NormalTok{, file_numbers, }\StringTok{".json"}\NormalTok{)}
\CommentTok{# At home}
\NormalTok{files <-}\StringTok{ }\KeywordTok{file.path}\NormalTok{(}\StringTok{"./JSON"}\NormalTok{, file_names)}
\CommentTok{# At uni}
\CommentTok{#files <- file.path("/course/Labs/Lab04/JSON", filenames)}
\NormalTok{json_data <-}\StringTok{ }\KeywordTok{do.call}\NormalTok{(rbind, }\KeywordTok{lapply}\NormalTok{(files, readJsonFileIntoDataframe))}
\NormalTok{json_data <-}\StringTok{ }\KeywordTok{subset}\NormalTok{(json_data, year }\OperatorTok{==}\StringTok{ "2018"} \OperatorTok{&}\StringTok{ }\NormalTok{vehicle_type }\OperatorTok{==}\StringTok{ "scooter"}\NormalTok{, }\DataTypeTok{select =} \KeywordTok{c}\NormalTok{(}\StringTok{"trip_duration"}\NormalTok{, }\StringTok{"trip_distance"}\NormalTok{))}
\KeywordTok{head}\NormalTok{(json_data)}
\end{Highlighting}
\end{Shaded}

\begin{verbatim}
##   trip_duration trip_distance
## 1           358           915
## 2           226           839
## 3           324          1206
## 4          1096             0
## 5           408          1144
## 6          1094          2631
\end{verbatim}

\begin{Shaded}
\begin{Highlighting}[]
\KeywordTok{dim}\NormalTok{(json_data)}
\end{Highlighting}
\end{Shaded}

\begin{verbatim}
## [1] 98817     2
\end{verbatim}

\subsection{MongoDB data}\label{mongodb-data}

\begin{Shaded}
\begin{Highlighting}[]
\KeywordTok{library}\NormalTok{(mongolite)}
\end{Highlighting}
\end{Shaded}

\begin{verbatim}
## Warning: package 'mongolite' was built under R version 3.6.1
\end{verbatim}

\begin{Shaded}
\begin{Highlighting}[]
\CommentTok{# At home}
\NormalTok{mongo_db <-}\StringTok{ }\KeywordTok{mongo}\NormalTok{(}\StringTok{"trips"}\NormalTok{, }\DataTypeTok{url =} \StringTok{"mongodb://localhost:27017/local"}\NormalTok{)}
\CommentTok{# At uni}
\CommentTok{#mongo_db <- mongo("trips")}
\NormalTok{mongo_db_data <-}\StringTok{ }\NormalTok{mongo_db}\OperatorTok{$}\KeywordTok{find}\NormalTok{(}
  \DataTypeTok{query =} \StringTok{'\{"vehicle_type": "scooter", "year": "2018"\}'}\NormalTok{,}
  \DataTypeTok{fields =} \StringTok{'\{"trip_duration": true, "trip_distance": true, "_id": false\}'}
\NormalTok{)}
\KeywordTok{head}\NormalTok{(mongo_db_data)}
\end{Highlighting}
\end{Shaded}

\begin{verbatim}
##   trip_duration trip_distance
## 1           358           915
## 2           226           839
## 3           324          1206
## 4          1096             0
## 5           408          1144
## 6          1094          2631
\end{verbatim}

\begin{Shaded}
\begin{Highlighting}[]
\KeywordTok{dim}\NormalTok{(mongo_db_data)}
\end{Highlighting}
\end{Shaded}

\begin{verbatim}
## [1] 98817     2
\end{verbatim}

\subsection{XML data}\label{xml-data}

\begin{Shaded}
\begin{Highlighting}[]
\KeywordTok{library}\NormalTok{(xml2)}
\end{Highlighting}
\end{Shaded}

\begin{verbatim}
## Warning: package 'xml2' was built under R version 3.6.1
\end{verbatim}

\begin{Shaded}
\begin{Highlighting}[]
\NormalTok{readXmlFileIntoDataframe <-}\StringTok{ }\ControlFlowTok{function}\NormalTok{(file_name) \{}
  \CommentTok{# Read in data}
\NormalTok{  xml_data <-}\StringTok{ }\KeywordTok{read_xml}\NormalTok{(file_name)}
\NormalTok{  trips <-}\StringTok{ }\KeywordTok{xml_find_all}\NormalTok{(xml_data, }\StringTok{"//row[vehicle_type = 'scooter'][year = 2018]"}\NormalTok{)}
\NormalTok{  trip_distance <-}\StringTok{ }\KeywordTok{as.numeric}\NormalTok{(}\KeywordTok{xml_text}\NormalTok{(}\KeywordTok{xml_find_first}\NormalTok{(trips, }\StringTok{"trip_distance"}\NormalTok{)))}
\NormalTok{  trip_duration <-}\StringTok{ }\KeywordTok{as.numeric}\NormalTok{(}\KeywordTok{xml_text}\NormalTok{(}\KeywordTok{xml_find_first}\NormalTok{(trips, }\StringTok{"trip_duration"}\NormalTok{)))}
  \KeywordTok{as.data.frame}\NormalTok{(}\KeywordTok{cbind}\NormalTok{(trip_duration, trip_distance))}
\NormalTok{\}}


\NormalTok{file_numbers <-}\StringTok{ }\KeywordTok{c}\NormalTok{(}\DecValTok{1}\OperatorTok{:}\DecValTok{10}\NormalTok{)}
\NormalTok{file_names <-}\StringTok{ }\KeywordTok{paste0}\NormalTok{(}\StringTok{"trips-"}\NormalTok{, file_numbers, }\StringTok{".xml"}\NormalTok{)}
\CommentTok{# At home}
\NormalTok{files <-}\StringTok{ }\KeywordTok{file.path}\NormalTok{(}\StringTok{"./XML"}\NormalTok{, file_names)}
\CommentTok{# At uni}
\CommentTok{#files <- file.path("/course/Labs/Lab04/JSON", filenames)}
\NormalTok{xml_data <-}\StringTok{ }\KeywordTok{do.call}\NormalTok{(rbind, }\KeywordTok{lapply}\NormalTok{(files, readXmlFileIntoDataframe))}
\KeywordTok{head}\NormalTok{(xml_data)}
\end{Highlighting}
\end{Shaded}

\begin{verbatim}
##   trip_duration trip_distance
## 1           358           915
## 2           226           839
## 3           324          1206
## 4          1096             0
## 5           408          1144
## 6          1094          2631
\end{verbatim}

\begin{Shaded}
\begin{Highlighting}[]
\KeywordTok{dim}\NormalTok{(xml_data)}
\end{Highlighting}
\end{Shaded}

\begin{verbatim}
## [1] 98817     2
\end{verbatim}

\section{Model the data and derive estimates of test error for 5
polynomial
models.}\label{model-the-data-and-derive-estimates-of-test-error-for-5-polynomial-models.}

Apply 10-fold cross-validation across 5 models with increasing
polynomial terms up to the a fifth order polynomial. We find that the
error begins to increase after the fourth order polynomial is added
(i.e.~with a fifth order polynomial term, the error increases).

\begin{Shaded}
\begin{Highlighting}[]
\NormalTok{## Cleanse our data}
\NormalTok{model_trips <-}\StringTok{ }\KeywordTok{subset}\NormalTok{(xml_data, }
\NormalTok{  trip_duration }\OperatorTok{>}\StringTok{ }\DecValTok{0} \OperatorTok{&}\StringTok{ }\NormalTok{trip_distance }\OperatorTok{>}\StringTok{ }\DecValTok{0}\NormalTok{)}
\NormalTok{trip_duration <-}\StringTok{ }\KeywordTok{log}\NormalTok{(model_trips}\OperatorTok{$}\NormalTok{trip_duration)}
\NormalTok{trip_distance <-}\StringTok{ }\KeywordTok{log}\NormalTok{(model_trips}\OperatorTok{$}\NormalTok{trip_distance)}
\KeywordTok{head}\NormalTok{(model_trips)}
\end{Highlighting}
\end{Shaded}

\begin{verbatim}
##   trip_duration trip_distance
## 1           358           915
## 2           226           839
## 3           324          1206
## 5           408          1144
## 6          1094          2631
## 7           705          1248
\end{verbatim}

\begin{Shaded}
\begin{Highlighting}[]
\NormalTok{## Define 10 splits}
\NormalTok{labels <-}\StringTok{ }\KeywordTok{rep}\NormalTok{(}\DecValTok{1}\OperatorTok{:}\DecValTok{10}\NormalTok{, }\DataTypeTok{length.out=}\KeywordTok{nrow}\NormalTok{(model_trips))}
\NormalTok{groups <-}\StringTok{ }\KeywordTok{sample}\NormalTok{(labels)}

\NormalTok{## Define our MSE function}
\NormalTok{mse <-}\StringTok{ }\ControlFlowTok{function}\NormalTok{(i, formula) \{}
\NormalTok{    test_set <-}\StringTok{ }\NormalTok{groups }\OperatorTok{==}\StringTok{ }\NormalTok{i}
\NormalTok{    train_set <-}\StringTok{ }\NormalTok{groups }\OperatorTok{!=}\StringTok{ }\NormalTok{i}
\NormalTok{    fit <-}\StringTok{ }\KeywordTok{lm}\NormalTok{(formula, }
              \KeywordTok{data.frame}\NormalTok{(}\DataTypeTok{x=}\NormalTok{trip_distance[train_set], }\DataTypeTok{y=}\NormalTok{trip_duration[train_set]))}
\NormalTok{    pred <-}\StringTok{ }\KeywordTok{predict}\NormalTok{(fit, }\KeywordTok{data.frame}\NormalTok{(}\DataTypeTok{x=}\NormalTok{trip_distance[test_set]))}
    \KeywordTok{mean}\NormalTok{((pred }\OperatorTok{-}\StringTok{ }\NormalTok{trip_duration[test_set])}\OperatorTok{^}\DecValTok{2}\NormalTok{)}
\NormalTok{\}   }

\NormalTok{## Train and test five models with increasing polynomial terms, look for lowest MSE value}
\NormalTok{mse_polynomial_model <-}\StringTok{ }\KeywordTok{data.frame}\NormalTok{(}\StringTok{"Order"}\NormalTok{ =}\StringTok{ "1"}\NormalTok{, }\StringTok{"MSE"}\NormalTok{ =}\StringTok{ }\KeywordTok{mean}\NormalTok{(}\KeywordTok{sapply}\NormalTok{(}\DecValTok{1}\OperatorTok{:}\DecValTok{10}\NormalTok{, mse, y }\OperatorTok{~}\StringTok{ }\NormalTok{x)))}
\NormalTok{mse_polynomial_model <-}\StringTok{ }\KeywordTok{rbind}\NormalTok{(mse_polynomial_model, }\KeywordTok{data.frame}\NormalTok{(}\StringTok{"Order"}\NormalTok{ =}\StringTok{ "2"}\NormalTok{, }\StringTok{"MSE"}\NormalTok{ =}\StringTok{ }\KeywordTok{mean}\NormalTok{(}\KeywordTok{sapply}\NormalTok{(}\DecValTok{1}\OperatorTok{:}\DecValTok{10}\NormalTok{, mse, y }\OperatorTok{~}\StringTok{ }\NormalTok{x }\OperatorTok{+}\StringTok{ }\KeywordTok{I}\NormalTok{(x}\OperatorTok{^}\DecValTok{2}\NormalTok{)))))}
\NormalTok{mse_polynomial_model <-}\StringTok{ }\KeywordTok{rbind}\NormalTok{(mse_polynomial_model, }\KeywordTok{data.frame}\NormalTok{(}\StringTok{"Order"}\NormalTok{ =}\StringTok{ "3"}\NormalTok{, }\StringTok{"MSE"}\NormalTok{ =}\StringTok{ }\KeywordTok{mean}\NormalTok{(}\KeywordTok{sapply}\NormalTok{(}\DecValTok{1}\OperatorTok{:}\DecValTok{10}\NormalTok{, mse, y }\OperatorTok{~}\StringTok{ }\NormalTok{x }\OperatorTok{+}\StringTok{ }\KeywordTok{I}\NormalTok{(x}\OperatorTok{^}\DecValTok{2}\NormalTok{) }\OperatorTok{+}\StringTok{ }\KeywordTok{I}\NormalTok{(x}\OperatorTok{^}\DecValTok{3}\NormalTok{)))))}
\NormalTok{mse_polynomial_model <-}\StringTok{ }\KeywordTok{rbind}\NormalTok{(mse_polynomial_model, }\KeywordTok{data.frame}\NormalTok{(}\StringTok{"Order"}\NormalTok{ =}\StringTok{ "4"}\NormalTok{, }\StringTok{"MSE"}\NormalTok{ =}\StringTok{ }\KeywordTok{mean}\NormalTok{(}\KeywordTok{sapply}\NormalTok{(}\DecValTok{1}\OperatorTok{:}\DecValTok{10}\NormalTok{, mse, y }\OperatorTok{~}\StringTok{ }\NormalTok{x }\OperatorTok{+}\StringTok{ }\KeywordTok{I}\NormalTok{(x}\OperatorTok{^}\DecValTok{2}\NormalTok{) }\OperatorTok{+}\StringTok{ }\KeywordTok{I}\NormalTok{(x}\OperatorTok{^}\DecValTok{3}\NormalTok{) }\OperatorTok{+}\StringTok{ }\KeywordTok{I}\NormalTok{(x}\OperatorTok{^}\DecValTok{4}\NormalTok{)))))}
\NormalTok{mse_polynomial_model <-}\StringTok{ }\KeywordTok{rbind}\NormalTok{(mse_polynomial_model, }\KeywordTok{data.frame}\NormalTok{(}\StringTok{"Order"}\NormalTok{ =}\StringTok{ "5"}\NormalTok{, }\StringTok{"MSE"}\NormalTok{ =}\StringTok{ }\KeywordTok{mean}\NormalTok{(}\KeywordTok{sapply}\NormalTok{(}\DecValTok{1}\OperatorTok{:}\DecValTok{10}\NormalTok{, mse, y }\OperatorTok{~}\StringTok{ }\NormalTok{x }\OperatorTok{+}\StringTok{ }\KeywordTok{I}\NormalTok{(x}\OperatorTok{^}\DecValTok{2}\NormalTok{) }\OperatorTok{+}\StringTok{ }\KeywordTok{I}\NormalTok{(x}\OperatorTok{^}\DecValTok{3}\NormalTok{) }\OperatorTok{+}\StringTok{ }\KeywordTok{I}\NormalTok{(x}\OperatorTok{^}\DecValTok{4}\NormalTok{) }\OperatorTok{+}\StringTok{ }\KeywordTok{I}\NormalTok{(x}\OperatorTok{^}\DecValTok{5}\NormalTok{)))))}
\NormalTok{mse_polynomial_model}
\end{Highlighting}
\end{Shaded}

\begin{verbatim}
##   Order       MSE
## 1     1 0.3927276
## 2     2 0.3367175
## 3     3 0.3144609
## 4     4 0.3051602
## 5     5 0.3053295
\end{verbatim}

\section{Conclusion summarising
analysis}\label{conclusion-summarising-analysis}

We can see from the following chart that the MSE flattens out at the
fourth order polynomial. Although it is not obvious from looking at the
chart, there is a slight increase in MSE at the fifth order polynomial.

\begin{Shaded}
\begin{Highlighting}[]
\KeywordTok{plot}\NormalTok{(mse_polynomial_model}\OperatorTok{$}\NormalTok{Order, mse_polynomial_model}\OperatorTok{$}\NormalTok{MSE, }\DataTypeTok{type=}\StringTok{"o"}\NormalTok{, }\DataTypeTok{xlab=}\StringTok{"Order of Polynomial"}\NormalTok{, }\DataTypeTok{ylab=}\StringTok{"MSE"}\NormalTok{, }\DataTypeTok{main=}\StringTok{"MSE by Order of Polynomial"}\NormalTok{)}
\end{Highlighting}
\end{Shaded}

\includegraphics{STATS769_2019_S2_bole001_lab04_files/figure-latex/unnamed-chunk-5-1.pdf}


\end{document}
